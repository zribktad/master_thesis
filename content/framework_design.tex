%!TEX root = ../main.tex

\chapter{Framework Design\label{chap:framework_design}}

...existing code...

\section{Framework Requirements}

The primary goal of the framework extension is to
enhance the existing unit testing capabilities by integrating advanced data preparation and handling mechanisms.
Unit testing is a critical practice in modern software development,
ensuring that individual components of an application behave as expected.
However, setting up and managing test data can often become a cumbersome process,
especially in complex systems where tests depend on intricate data structures,
external services, or specific state configurations.
\subsection*{Functional Requirements}

\begin{itemize}
    \item \textbf{Custom Data Preparation Attributes:}
    {\sloppy
    \begin{itemize}
        \item Implement custom attributes to facilitate specific data setup and teardown processes for each test case.
        \item Attributes should allow dynamic injection of test data and enable conditional test execution based on data state.
    \end{itemize}
    }
    \item \textbf{Automated Test Data Handling:}
    \begin{itemize}
        \item Develop a \texttt{TestDataHandler} class to automate the execution of data preparation methods before and after tests, ensuring a consistent test environment.
        \item Ensure that both simple and complex test data structures can be seamlessly prepared and injected.
    \end{itemize}
    
    \item \textbf{Data Preparation Store:}
    \begin{itemize}
        \item Create a \texttt{TestDataPreparationStore} to maintain mappings of data preparation instances associated with test methods, enabling efficient data management.
        \item Implement thread-safe access for concurrent test executions in multi-threaded environments.
    \end{itemize}
    
    \item \textbf{Integration with NUnit Test Lifecycle:}
    \begin{itemize}
        \item Leverage NUnit's \texttt{ITestAction} interface to integrate custom behaviors into the test execution lifecycle, using methods like \texttt{BeforeTest} and \texttt{AfterTest}.
        \item Ensure custom lifecycle hooks are compatible with NUnit's parallel test execution model.
    \end{itemize}
    
    \item \textbf{Attribute Count Tracking:}
    \begin{itemize}
        \item Implement a \texttt{TestAttributeCountStore} to track the execution of custom attributes and ensure all data preparations are completed before test execution.
        \item Prevent redundancy in data preparation execution for tests using multiple attributes.
    \end{itemize}
    
    \item \textbf{Service Provider Utilization:}
    \begin{itemize}
        \item Use a service provider pattern to manage dependencies and services required for data preparation.
        \item Support integration with dependency injection frameworks.
    \end{itemize}
\end{itemize}

\subsection*{Non-Functional Requirements}

\begin{itemize}
    \item \textbf{Performance Efficiency:}
    \begin{itemize}
        \item Ensure minimal overhead is introduced during test execution to maintain optimal performance.
        \item Optimize data preparation logic to minimize unnecessary computation or resource usage.
    \end{itemize}
    
    \item \textbf{Modularity and Extensibility:}
    \begin{itemize}
        \item Design the framework with a modular architecture to facilitate easy maintenance and future enhancements.
        \item Provide extension points for developers to customize or extend attributes and lifecycle handling.
    \end{itemize}
    
    \item \textbf{Compliance with Coding Standards:}
    \begin{itemize}
        \item Adhere to coding best practices and standards to ensure code quality and readability.
        \item Follow SOLID principles for maintainable and testable code.
    \end{itemize}
    
    \item \textbf{Seamless NUnit Integration:}
    \begin{itemize}
        \item The extension should integrate smoothly with the existing NUnit framework without disrupting current testing workflows.
        \item Support legacy NUnit test cases alongside the new extension.
    \end{itemize}
\end{itemize}

% Your content here.

\section{Framework Architecture}

% Your content here.

\section{Extending NUnit}

% Your content here.